\makeglossaries

\newglossaryentry{outer product}
{
        name=outer product,
        description={The matrix produced by the product of a column vector and its transpose, i.e., \(\mathbf{u}\mathbf{v}^*\)}
}

\newglossaryentry{upper triangular matrix}
{
        name=upper triangular matrix,
        description={A matrix where all entries below the diagonal are zero, i.e., \[U = \begin{bmatrix}
                                        u_{1,1} & u_{1,2} & u_{1,3} & \ldots & u_{1,n}   \\
                                        0       & u_{2,2} & u_{2,3} & \ldots & u_{2,n}   \\
                                        0       & 0       & \ddots  & \ddots & \vdots    \\
                                        0       & 0       & 0       & \ddots & u_{n-1,n} \\
                                        0       & 0       & 0       & 0      & u_{n,n}
                                \end{bmatrix}\]}
}

\newglossaryentry{lower triangular matrix}
{
        name=lower triangular matrix,
        description={A matrix where all entries above the diagonal are zero, i.e., \[L = \begin{bmatrix}
                                        \ell_{1,1} & 0          & 0      & 0            & 0          \\
                                        \ell_{2,1} & \ell_{2,2} & 0      & 0            & 0          \\
                                        \ell_{3,1} & \ell_{3,2} & \ddots & 0            & 0          \\
                                        \vdots     & \vdots     & \ddots & \ddots       & 0          \\
                                        \ell_{n,1} & \ell_{n,2} & \ldots & \ell_{n,n-1} & \ell_{n,n}
                                \end{bmatrix}\]}
}

\newglossaryentry{range}
{
        name=range,
        description={The set of all vectors that can be expressed as \(A\mathbf{x}\) for some \(\mathbf{x}\).}
}

\newglossaryentry{nullspace}
{
        name=nullspace,
        description={The set of all vectors that satisfy \(A\mathbf{x}=\mathbf{0}\) (see also: kernel).}
}

\newglossaryentry{kernel}
{
        name=kernel,
        description={The set of all vectors that satisfy \(A\mathbf{x}=\mathbf{0}\) (see also: nullspace).}
}

\newglossaryentry{rank}
{
        name=rank,
        description={The dimension of the vector space spanned by a matrix.}
}

\newglossaryentry{column rank}
{
        name=column rank,
        description={The dimension of the vector space spanned by the columns of a matrix.}
}

\newglossaryentry{row rank}
{
        name=row rank,
        description={The dimension of the vector space spanned by the rows of a matrix.}
}

\newglossaryentry{full rank}
{
        name=full rank,
        description={An \(m\times n\) matrix has \textbf{full rank} when its rank is equal to \(\min\left(m, n\right)\).}
}

\newglossaryentry{singular matrix}
{
        name=singular matrix,
        description={A matrix that has no inverse. Equivalently, a matrix with 0 determinant.}
}

\newglossaryentry{non-singular matrix}
{
        name=non-singular matrix,
        description={A matrix that has an inverse. Equivalently, a matrix with non-zero determinant.}
}

\newglossaryentry{symmetric}
{
        name=symmetric,
        description={A matrix \(A\) such that \(A=A^\text{T}\).}
}

\newglossaryentry{inner product}
{
        name=inner product,
        description={The \textbf{inner product} of \(\mathbf{u}\) and \(\mathbf{v}\) is \(\mathbf{u}^{\text{T}}\mathbf{v}\). Also called the dot product.}
}

\newglossaryentry{orthogonal vectors}
{
        name=orthogonal vectors,
        description={Two vectors \(\mathbf{u}\) and \(\mathbf{v}\) are \textbf{orthogonal} if their inner product is 0, i.e., if \(\mathbf{u}^{\text{T}}\mathbf{v}=0\).}
}

\newglossaryentry{orthogonal matrix}
{
        name=orthogonal matrix,
        description={A real square matrix \(A\) is orthogonal if \(AA^{\text{T}}=I\), i.e., \(A^{-1}=A^{\text{T}}\).}
}

\newglossaryentry{unitary matrix}
{
        name=unitary matrix,
        description={A square matrix \(Q\) is unitary if \(QQ^*=I\), i.e., \(Q^{-1}=Q^*\).}
}

\newglossaryentry{norm}
{
        name=norm,
        description={A norm is a real-valued function \(\norm{.}\) such that
                        \begin{enumerate}
                                \item \(\norm{\mathbf{x} + \mathbf{y}} \leq \norm{\mathbf{x}} + \norm{\mathbf{y}}\) (triangle inequality)
                                \item \(\norm{c\mathbf{x}}=\abs{c}\norm{\mathbf{x}}\)
                                \item \(\norm{\mathbf{x}}=0\) if and only if \(\mathbf{x}=\mathbf{0}\)
                        \end{enumerate}
                }
}

\newglossaryentry{p-norm}
{
        name={\ensuremath{p}-norm},
        description={\[\norm{\mathbf{x}}_p=\left(\sum_{i=1}^n\abs{x_i}^p\right)^{\frac{1}{p}}\]
        },
        sort=p-norm
}

\newglossaryentry{infinity-norm}
{
        name={\ensuremath{\infty}-norm},
        description={\[\norm{\mathbf{x}}_{\infty}=\max_{1\leq i\leq m}\abs{x_i}\]
                },
        sort=infinity-norm
}

\newglossaryentry{induced norm}
{
        name=induced norm,
        description={A set of vectors \(\mathbf{x_1}\), \(\mathbf{x_2}\), \(\hdots\), \(\mathbf{x_n}\) is linearly independent of the only solution to the equation \[c_1\mathbf{x_1}+c_2\mathbf{x_2}+\hdots+c_n\]}
        description={Given a norm \(\norm{.}\) on a vector space, the \textbf{induced norm} on matrices is
        \begin{align*}\norm{A}
                & =\sup\left\{\frac{\norm{A\mathbf{x}}}{\norm{\mathbf{x}}}:\norm{\mathbf{x}}\neq\mathbf{0}\right\} \\
                & =\sup\left\{\norm{A\mathbf{x}}:\norm{\mathbf{x}}=1\right\}
        \end{align*}}
}

\newglossaryentry{frobenius norm}
{
        name=Frobenius norm,
        description={\[\norm{\mathbf{x}}_F=\left(\sum_{i=1}^m\sum_{j=1}^n\abs{a_{ij}}^2\right)^{\frac{1}{2}}\]}
}

\newglossaryentry{singular value decomposition}
{
        name=Singular Value Decomposition,
        description={For any \(m\times n\) matrix \(A\), a \textbf{Singular Value Decomposition} (SVD) of \(A\) is a factorization \[A=U\Sigma V^*\] where
                        \begin{itemize}
                                \item \(U\) is an \(m\times m\) unitary matrix
                                \item \(V\) is an \(n\times n\) unitary matrix
                                \item \(\Sigma\) is an \(m\times n\) diagonal matrix with non-negative real numbers on the digonal
                        \end{itemize}
                        The columns of \(U\) and \(V\) are called the left-singular and right-singular vectors, respectively, and the diagonal entries \(\sigma_{ii}\) of \(\Sigma\) are known as the \textbf{singular values}.}
}

\newglossaryentry{singular value}
{
        name=singular value,
        description={The diagonal entries of the matrix \(\Sigma\) in the Singular Value Decomposition.}
}

\newglossaryentry{eigenvalue}
{
        name=eigenvalue,
        description={A scalar \(\lambda\) such that \(A\mathbf{x}=\lambda\mathbf{x}\) for non-zero \(\mathbf{x}\).}
}

\newglossaryentry{eigenvector}
{
        name=eigenvector,
        description={A non-zero vector \(\mathbf{x}\) such that \(A\mathbf{x}=\lambda\mathbf{x}\) for some scalar \(\lambda\).}
}

\newglossaryentry{dimension}
{
        name=dimension,
        description={The size of a basis of a vector space. Equivalently, the greatest number of linearly independent vectors, or the least number of vectors which spans the vector space (see also: rank).}
}

\newglossaryentry{span}
{
        name=span,
        description={The set of all linear combinations of the columns of a matrix.}
}

\newglossaryentry{linearly independent}
{
        name=linearly independent,
        description={A set of vectors \(\mathbf{x_1}\), \(\mathbf{x_2}\), \(\hdots\), \(\mathbf{x_n}\) is linearly independent of the only solution to the equation \[c_1\mathbf{x_1}+c_2\mathbf{x_2}+\hdots+c_n\mathbf{x_n}=\mathbf{0}\] is \[c_1=c_2=\hdots=c_n=0\]}
}

\newglossaryentry{basis}
{
        name=basis,
        description={A \textbf{basis} of a vector space is a set of linearly independent vectors that spans the entire space.}
}