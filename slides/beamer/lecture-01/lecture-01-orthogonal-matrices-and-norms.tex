\documentclass[10pt]{beamer}
\usefonttheme[onlymath]{serif}
\usepackage[utf8]{inputenc}
% \AtBeginSection[]
% {
%   \begin{frame}
%     \frametitle{Table of Contents}
%     \tableofcontents[currentsection]
%   \end{frame}
% }
\usetheme{Berkeley}
\usepackage{mleftright}
\mleftright

\usepackage{tikzsymbols}

\newcommand{\abs}[1]{\left\lvert#1\right\rvert}
\newcommand{\norm}[1]{\left\lVert#1\right\rVert}
\renewcommand{\vec}[1]{\boldsymbol{#1}}
\newcommand{\proj}[2]{\operatorname{proj}_{#2}\left(#1\right)}
\newcommand{\dotprod}[2]{\left\langle#1,#2\right\rangle}
\newcommand{\compconj}[1]{\overline{#1}}
\newcommand{\T}{\mathrm{T}}
\newcommand{\set}[1]{\left\{#1\right\}}
\newcommand{\bigO}[1]{\mathcal{O}\left(#1\right)}
%Information to be included in the title page:

\title[COSC6364 Adv. Numerical Analysis] %optional
{COSC 6364 Adv. Numerical Analysis}

\subtitle{Lecture 1: Matrix-Vector Multiplication; Orthogonal Vectors and Matrices}
\author[Wu, Panruo] % (optional, for multiple authors)
{P. ~Wu\inst{1}}

\institute[VFU] % (optional)
{
  \inst{1}%
  Computer Science\\
  University of Houston
  % \and
  % \inst{2}%
  % Faculty of Chemistry\\
  % Very Famous University
}

\date[Spring 2020] % (optional)
{Spring 2020\\To accompany the Numerical Linear Algebra book, Lectures 1 and 2.}

\logo{\includegraphics[height=0.5cm]{../nsm-computer-science-primary.png}}

\begin{document}

\frame{\titlepage}

\begin{frame}
  \frametitle{Table of Contents}
  \tableofcontents
\end{frame}

\section{Matrix Vector Multiplication}
\begin{frame}
  \frametitle{Matrix Vector Multiplication}
  \begin{itemize}
    \item <2-> New interpretation that is essential for numerical linear algebra: \alert{if \(\vec{b} = A\vec{x}\), then \(\vec{b}\) is the linear combination of the columns of \(A\).}
    \begin{itemize}
      \item <3-> Let \(\vec{a}_i\) denote the \(i^{\text{th}}\) column of \(A\), \(m-\)vector. Then we can write
        \onslide<4->{\[\vec{b} = X = \sum_{i=1}^nx_i\vec{a}_i\]}
        \onslide<5->{The equation can be displayed schematically as follows:}
        \begin{align*}\onslide<6->{\begin{bmatrix}b\end{bmatrix} 
                  &= \begin{bmatrix}\vec{a}_1 & \vec{a}_2 & \hdots & \vec{a}_n\end{bmatrix} \begin{bmatrix}x_1\\x_2\\\vdots\\x_n\end{bmatrix}}\\
                  \onslide<7->{&= x_1\vec{a}_1 + x_2\vec{a}_2 + \hdots + x_n\vec{a}_n}
                \end{align*}
    \end{itemize}
  \end{itemize}
\end{frame}

\begin{frame}
  \frametitle{Matrix Vector Multiplication}
  \begin{itemize}
   \item <2-> Mathematician: \(A\vec{x} = \vec{b}\) means matrix \(A\) (representing linear transformation) acts on vector \(\vec{x}\) to produce \(\vec{b}\).
   \item <3-> Numerical analyst: \(A\vec{x} = \vec{b}\) means vector \(\vec{x}\) acts on \(A\) to produce \(\vec{b}\) (as linear combination of columns of \(A\)).
   \item <4->Function can be seen as an \alert{infinite dimensional vector}. Matrix A’s columns can be function
  \end{itemize}
\end{frame}

\section{Matrix Times a Matrix}
\begin{frame}
  \frametitle{Matrix Times a Matrix}
  \begin{itemize}
    \item <2-> For a matrix product \(B = AC\), each column of \(B\) is a linear combination of the columns of \(A\).
    \item <3-> To see this, write:
    \begin{align*}\onslide<4->{AC 
      &= \underbrace{\begin{bmatrix}a_{11} & a_{12} & \hdots & a_{1n}\\a_{21} & a_{22} & \hdots & a_{2n}\\\vdots & \vdots & \ddots & \vdots\\a_{m1} & a_{m2} & \hdots & a_{mn}\end{bmatrix}}_{m\times n}\underbrace{\begin{bmatrix}c_{11} & c_{12} & \hdots & c_{1k}\\c_{21} & c_{22} & \hdots & a_{2k}\\\vdots & \vdots & \ddots & \vdots\\c_{n1} & c_{n2} & \hdots & c_{nk}\end{bmatrix}}_{n\times k}}\\
    \onslide<5->{B  &= \underbrace{\begin{bmatrix}\mathbf{b_1} \mid \mathbf{b_2} \mid \hdots \mid \mathbf{b_k}\end{bmatrix}}_{m\times k}}
    \end{align*} 
  \end{itemize}
\end{frame}

\begin{frame}
  \frametitle{Matrix Times a Matrix}
  Then column \(\mathbf{b_i}\) is just 
\begin{align*}\onslide<2->{\mathbf{b_i}
    &= \begin{bmatrix}a_{11}c_{1i} + a_{12}c_{2i} + \hdots + a_{1n}c_{ni}\\a_{21}c_{1i} + a_{22}c_{2i} + \hdots + a_{2n}c_{ni}\\\vdots\\a_{m1}c_{1i} + a_{m2}c_{2i} + \hdots + a_{mn}c_{ni}\end{bmatrix}}\\
    &\onslide<3->{= c_{1i}\begin{bmatrix}a_{11}\\a_{21}\\\vdots\\a_{m1}\end{bmatrix} + c_{2i}\begin{bmatrix}a_{12}\\a_{22}\\\vdots\\a_{m2}\end{bmatrix} + \hdots + c_{ni}\begin{bmatrix}a_{1n}\\a_{2n}\\\vdots\\a_{mn}\end{bmatrix}}\\
    &\onslide<4->{= c_{1i}\mathbf{a_1} + c_{2i}\mathbf{a_2} + \hdots + c_{ni}\mathbf{a_n}}
\end{align*}
\end{frame}

\section{Examples}
\subsection{Outer Product}
\begin{frame}
  \frametitle{Outer Product}
\begin{align*}\onslide<2->{\vec{u}\vec{v}^* 
  &= \vec{u}\begin{bmatrix}\compconj{v_1} & \compconj{v_2} & \hdots & \compconj{v_n}\end{bmatrix}}\\
  &\onslide<3->{= \begin{bmatrix}\compconj{v_1}\vec{u} & \compconj{v_2}\vec{u} & \hdots & \compconj{v_n}\vec{u}\end{bmatrix}}\\
  &\onslide<4->{= \begin{bmatrix}u_1\compconj{v_1} & u_1\compconj{v_2} & \hdots & u_1\compconj{v_n}\\u_2\compconj{v_1} & u_2\compconj{v_2} & \hdots & u_2\compconj{v_n}\\\vdots & \vdots & \ddots & \vdots\\u_n\compconj{v_1} & u_n\compconj{v_2} & \hdots & u_n\compconj{v_n}\end{bmatrix}}
\end{align*}
\end{frame}
\end{document}
