\section*{Quiz 1}\label{sec:quiz-1}
\begin{enumerate}
\item Given a matrix that is both triangular and unitary, is it non-diagonal?
\begin{solution}
    A triangular, unitary matrix must be diagonal. To see this, consider an upper-triangular, normal matrix\footnote{A normal matrix is one which satisfies \(AA^*=A^*A\). Clearly, every unitary matrix is normal.}, \(A\). Write \begin{align*}A&=\begin{bmatrix}a_{11} & a_{12} & a_{13} & \hdots & a_{1n}\\0 & a_{22} & a_{23} & \hdots & a_{2n}\\0 & 0 & a_{33} & \hdots & a_{3n}\\\vdots & \vdots & \vdots & \ddots & \vdots\\0 & 0 & 0 & \hdots & a_{nn}\end{bmatrix}\\A^*&=\begin{bmatrix}\overline{a_{11}} & 0 & 0 & \hdots & 0\\\overline{a_{12}} & \overline{a_{22}} & 0 & \hdots & 0\\\overline{a_{13}} & \overline{a_{23}} & \overline{a_{33}} & \hdots & 0\\\vdots & \vdots & \vdots & \ddots & \vdots\\\overline{a_{1n}} & \overline{a_{2n}} & \overline{a_{3n}} & \hdots & \overline{a_{nn}}\end{bmatrix}\end{align*}

    Consider the \(1^{\text{st}}\) diagonal element of \(AA^*\), writen explicitly:

    \[(AA^*)_{11}=a_{11}\overline{a_{11}}+a_{12}\overline{a_{12}}+\hdots + a_{1n}\overline{a_{1n}}\]

    And similarly for \(A^*A\):

    \[(A^*A)_{11}=\overline{a_{11}}a_{11}\] These two values must be equal, forcing \[a_{12}\overline{a_{12}}+\hdots + a_{1n}\overline{a_{1n}}=0\] However, \(z\overline{z}=|z|\) is strictly non-negative, hence these values must be identically 0. In particular, this means the first row of \(A\) is 

    \[\begin{bmatrix}a_{11} & 0 & 0 & \hdots & 0\end{bmatrix}\] The same argument applies for each row of the matrix \(A\). 
        
        If \(A\) is lower-triangular, then \(B=A^*\) is upper-triangular, and \(B\) is diagonal, by the above argument, hence \(A\) is diagonal.
\end{solution}
\item Can the absolute value of an eigenvalue of a unitary matrix be 1?
\begin{solution}
    Clearly, the answer is yes. Take \(A=I_n\) and note that it has characteristic equation \((1-\lambda)^n=0\), which has eigenvalues of 1. However, the stronger result is that \textit{all} eigenvalues of a unitary matrix have modulus 1:
    
Consider some unitary matrix \(A\), i.e., \(A\) satisfies \(AA^*=A^*A=I\), and any eigenvalue, \(\lambda\). We have \[A\mathbf{x}=\lambda\mathbf{x}\] for some vector \(\mathbf{x}\). Taking the conjugate transpose of both sides gives \[\mathbf{x}^*A^*=\overline{\lambda}\mathbf{x}^*\] Multiplying these equations yields \begin{align*}\left(\mathbf{x}^*A^*\right)\left(A\mathbf{x}\right)&=\left(\overline{\lambda}\mathbf{x}^*\right)\left(\lambda\mathbf{x}\right)\\\mathbf{x}^*\left(A^*A\right)\mathbf{x}&=\lambda\overline{\lambda}\mathbf{x}^*\mathbf{x}\\\mathbf{x}^*\mathbf{x}&=\lambda\overline{\lambda}\mathbf{x}^*\mathbf{x}\\\end{align*}This forces \(\lambda\overline{\lambda}=\abs{\lambda}^2=1\).
\end{solution}
\item If \(W\) is an arbitrary nonsingular matrix, then is the function \(\norm{.}_W\) defined by \(\norm{\mathbf{x}}_W=\norm{W\mathbf{x}}\) (weighted norm) a vector norm?
\begin{solution}
    In order for \(\norm{.}_W\) to be a vector norm, it must satisfy:
    \begin{enumerate}[1.]
        \item \(\norm{\mathbf{u} + \mathbf{v}}_W \leq \norm{\mathbf{u}}_W + \norm{\mathbf{v}}_W\) (triangle inequality)
        \item \(\norm{c\mathbf{u}}_W=\abs{c}\norm{\mathbf{u}}_W\) (scalable/homogenous)
        \item if \(\norm{\mathbf{u}}_W = 0\) then \(\mathbf{u}=\mathbf{0}\) (positivity)
    \end{enumerate}
    2 and 3 are obvious. To see 1, note that 
    \begin{align*}\norm{\mathbf{u} + \mathbf{v}}_W
        &=    \norm{W\left(\mathbf{u} + \mathbf{v}\right)}\\
        &=    \norm{W\mathbf{u} + W\mathbf{v}}\\
        &\leq \norm{W\mathbf{u}} + \norm{W\mathbf{v}}\text{ by the triangle inequality}\\
        &=    \norm{\mathbf{u}}_W + \norm{\mathbf{v}}_W
    \end{align*}
    Thus, \(\norm{.}_W\) is a vector norm.
\end{solution}
\item If \(E\) is an outer product \(E=\mathbf{u}\mathbf{v}^*\), then \(\norm{E}_2 = \norm{\mathbf{u}}_2\norm{\mathbf{v}}_2\). Is the same true for the Frobenius norm, i.e., \(\norm{E}_F=\norm{\mathbf{u}}_F \norm{\mathbf{v}}_F\)?
\begin{solution}
Write \(\mathbf{u}=\begin{bmatrix}u_1 \\ u_2 \\ \vdots \\ u_n\end{bmatrix}\) and \(\mathbf{v}=\begin{bmatrix}v_1 \\ v_2 \\ \vdots \\ v_n\end{bmatrix}\). Then \begin{align*}E
    &= \mathbf{u}\mathbf{v}^*\\
    &= \begin{bmatrix}u_1 \\ u_2 \\ \vdots \\ u_n\end{bmatrix} \begin{bmatrix}\overline{v_1} & \overline{v_2} & \hdots & \overline{v_n}\end{bmatrix}\\
    &= \begin{bmatrix}u_1\overline{v_1} & u_1\overline{v_2} & \hdots & u_1\overline{v_n}\\u_2\overline{v_1} & u_2\overline{v_2} & \hdots & u_2\overline{v_n}\\\vdots & \vdots & \ddots & \vdots \\ u_n\overline{v_1} & u_n\overline{v_2} & \hdots & u_n\overline{v_n}\end{bmatrix}
\end{align*}
We therefore have \begin{align*}\norm{E}_F&=\sqrt{\sum_{i=1}^n\sum_{j=1}^n|u_i\overline{v_j}|^2}\\&=\sqrt{\sum_{i=1}^n\sum_{j=1}^n|u_i|^2|\overline{v_j}|^2}\\&=\sqrt{\sum_{i=1}^n\sum_{j=1}^n|u_i|^2|v_j|^2}\end{align*} And further
\begin{align*}\norm{\mathbf{u}}_F \norm{\mathbf{v}}_F
    &= \left(\sqrt{\sum_{i=1}^n|u_i|^2}\right)\left(\sqrt{\sum_{i=1}^n|v_i|^2}\right)\\
    &= \sqrt{\left(\sum_{i=1}^n|u_i|^2\right)\left(\sum_{i=1}^n|v_i|^2\right)}\\
    &= \sqrt{\sum_{i=1}^n\sum_{j=1}^n|u_i|^2|v_j|^2}\\
    &= \norm{E}_F
\end{align*}
\end{solution}
\end{enumerate}
